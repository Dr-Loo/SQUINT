\documentclass[11pt]{article}

% --- Packages ---
\usepackage[a4paper,margin=1in]{geometry}
\usepackage{amsmath,amssymb,amsfonts}
\usepackage{bm}
\usepackage{graphicx}
\usepackage{booktabs}
\usepackage{hyperref}
\usepackage{siunitx}
\usepackage{enumitem}
\usepackage{xcolor}
\usepackage{listings}
\usepackage{cite}

% --- Links ---
\hypersetup{
  colorlinks=true,
  linkcolor=black,
  citecolor=blue!50!black,
  urlcolor=blue!50!black
}

% --- Macros ---
\newcommand{\Teff}{T_{\mathrm{eff}}}
\newcommand{\Qtop}{Q_{\mathrm{top}}}
\newcommand{\Phif}{\Phi}
\newcommand{\SQUINT}{\textsc{SQUINT}}
\newcommand{\qua}{\textsc{QUA}}
\newcommand{\guppy}{\textsc{Guppy}}

% --- Title ---
\title{\bf SQUINT: Semantic-Field-Aware Quantum Compilation with Defect Tracking and Floquet Scheduling}
\author{Dr. Loo}
\date{\today}

\begin{document}
\maketitle

\begin{abstract}
\noindent
We present \SQUINT, a domain-aware quantum compiler that embeds \emph{semantic fields} ($\Phif$), \emph{topological-defect tracking}, and \emph{thermodynamic diagnostics} directly into compilation. Programs express physical intent via overlay directives---coherence length, lattice path length, field-dependent damping, and braid handles---and may specify Floquet parameters (period, duty, phase step) to synthesize periodic control sequences automatically. \SQUINT\ validates constraints, performs multi-domain analysis, and schedules control pulses, emitting QUA-like pulse text together with a machine-readable timeline. A lightweight execution layer simulates field evolution, defect hysteresis, phantom-heat proxies, and Floquet cycles to recover observables such as $\Teff$ and corrected readouts. In v0.1 we implement: (i) overlay-driven wait insertion and path-length enforcement; (ii) Floquet expansion of a single control into cycle trains with per-cycle logging; (iii) JSON timeline emission alongside QUA-like output; and (iv) integrated simulation and visualization. \SQUINT\ offers a reproducible path from physical intent to pulse-level artifacts, bridging semantics, topology, and control.
\end{abstract}

\vspace{1ex}
\noindent\textbf{Keywords:} quantum compilation; semantic fields; topological defects; Floquet control; overlays; pulse-level IR; timeline logging.

% =========================================================
\section{Introduction}
Gate-level quantum compilers excel at algebraic circuit identities and device mapping, but largely ignore \emph{semantic structure} (field meaning), \emph{topological memory} (defects/braids), and \emph{thermodynamic diagnostics} (heating, effective temperature). Emerging hardware---superconducting arrays with tunable couplers, topological platforms with anyonic defects, and analog, Floquet-driven devices---exposes these phenomena to the programmer.

We address this gap with \SQUINT, a compiler where physical intent appears at the source level via \emph{overlays}. Overlays express constraints (e.g., coherence length $\ge$ \SI{120}{\nano\second}, path length $\le 2$), couplings to a semantic field $\Phif$ (for damping), and optional Floquet schedules. The compiler validates overlays, schedules pulses accordingly, and emits QUA-like text plus a detailed timeline to support downstream control stacks and analysis.

\paragraph{Contributions.}
\begin{itemize}[leftmargin=1.3em]
  \item \textbf{Overlay-aware language \& IR.} A minimal syntax that carries coherence/path/damping/braid and Floquet metadata; parsers and IR that preserve this semantics.
  \item \textbf{Constraint validation \& scheduling.} Automatic wait insertion to satisfy coherence, path-length checks on lattice embeddings, and cycle-train expansion for Floquet drives.
  \item \textbf{Pulse-level emission \& timeline.} QUA-like text with structured comments for semantic/braid ops and a JSON timeline for reproducibility and tooling.
  \item \textbf{Execution layer \& observables.} Toy yet deterministic simulation of $\Phif$ and defect density/hysteresis yielding $\Teff$ and phantom-heat proxies, plus a visualizer.
\end{itemize}

% =========================================================
\section{Language Design}
\label{sec:language}
The language provides a single \texttt{workspace} with qubits, a rectangular lattice, semantic and defect fields, and a \texttt{kernel} over that workspace.

\subsection{Core Syntax}
\begin{lstlisting}[language=C, basicstyle=\ttfamily\small, frame=single]
workspace Chip {
  qubits q[4];
  lattice L(2,2) attach q;
  semantic_field Phi : scalar on L;
  defect_field D     : defects on L { kinetics = "overdamped"; };
}

kernel CalibratedEPR on Chip {
  initialize Phi = constant(0.4);
  nucleate D at {(0,0),(1,1)};
  evolve   D with rule braid_exchange(rate=0.7, conserve_Qtop=true);

  ctrl rx q[0] angle = π/2 with overlay { coherence_len >= 80ns };
  ctrl cz q[0], q[1] with overlay {
    coherence_len >= 120ns, path_len <= 2,
    floquet_period = 50ns, cycles = 8, duty = 0.4, phase_step = 12deg
  };

  quench δQ_top = inject(D, amount=0.02);
  observe T_eff into Te with corrections { defects=D, field=Phi };
  hysteresis_trace(D, window=3);

  measure q[0] -> m0;
  measure q[1] -> m1;
  return { Te, corr = m0 ⊕ m1 };
}
\end{lstlisting}

\subsection{Overlays}
Overlays attach physical intent to controls:
\begin{itemize}[leftmargin=1.3em]
  \item \textbf{Coherence:} \verb|coherence_len >= Nns| inserts \verb|wait(N)|.
  \item \textbf{Path length:} \verb|path_len <= k| checks Manhattan distance under the current lattice.
  \item \textbf{Damping:} \verb|damping = η(Φ=Phi)| validates the referenced field and couples semantics to control.
  \item \textbf{Braid handle:} \verb|braid = D| requires a declared defect field.
  \item \textbf{Floquet:} \verb|floquet_period, cycles, duty, phase_step| expand a single control into a cycle train with per-cycle logging.
\end{itemize}

% =========================================================
\section{Compiler Architecture}
\label{sec:compiler}
\paragraph{Classification.} Statements are categorized as \emph{quantum} (\texttt{ctrl}, \texttt{measure}), \emph{semantic} (\texttt{initialize}, \texttt{observe}, \texttt{transport}, \texttt{return}), or \emph{braid} (\texttt{nucleate}, \texttt{evolve}, \texttt{quench}, \texttt{hysteresis\_trace}).

\paragraph{Overlay Solver.} For each control, \SQUINT\ validates and normalizes overlay keys. Coherence constraints insert waits; path-length checks compute lattice distances; damping/braid entries validate references. Failures are informative; in \emph{strict} mode, they become hard errors.

\paragraph{Floquet Scheduler.} If present, the scheduler produces per-cycle emissions (gate then OFF window), annotating the timeline as \texttt{gate@floquet} with a \texttt{cycle} index.

\paragraph{Emission.} The QUA-like emitter produces a compact program with \verb|wait(...)| insertions and structured comments for semantic/braid ops. A JSON log captures the timeline and events for tooling and visualization.

% =========================================================
\section{Execution Semantics}
\label{sec:exec}
We provide a small, deterministic simulator to drive examples and visualizations. The semantic field $\Phif$ evolves via a smoothed diffusion/decay step; defects $D$ maintain coordinates, density, and phase; a quench reduces density while preserving handles. A phantom-heat proxy is defined as the $\ell_2$ norm of $\Delta \Phif$ across frames. Observables include $\Teff$ (base + defect + field terms) and measurement corrections.

% =========================================================
\section{Observables and Thermodynamics}
\label{sec:observables}
\paragraph{Effective temperature $\Teff$.} Computed from base field level with additive corrections from defect and field terms.
\paragraph{Topological charge proxy $\Qtop$.} Derived from declared braid handles and defect evolution events.
\paragraph{Hysteresis.} The \texttt{hysteresis\_trace} records a short window of defect density; range and drift provide a semantic memory diagnostic.
\paragraph{Cycle-resolved metrics.} When Floquet is active, traces are indexed by cycle for alignment with on/off windows.

% =========================================================
\section{Visualization and Tooling}
\label{sec:viz}
The \emph{Floquet Visualizer} animates $\Phif$ on the lattice with defect overlays and plots $\Teff$, phantom heat, and hysteresis. Inputs are the emitted \verb|*.log.json| and \verb|*.sim.json|. The web playgrounds provide an interactive front-end to generate overlay blocks and preview cycle trains.

% =========================================================
\section{Evaluation (v0.1)}
\label{sec:evaluation}
\paragraph{Benchmarks.} We include three examples (\texttt{basic}, \texttt{floquet}, \texttt{semantic}). For each, \SQUINT\ compiles to QUA-like text, writes a timeline, and (optionally) produces simulation traces.
\paragraph{Reproducibility.} Each compile produces human-readable text and machine logs; the visualizer consumes the same logs to regenerate figures.
\paragraph{Ablations.} Removing overlays yields minimal schedules; enabling overlays triggers wait insertion, path validation, and Floquet cycle trains.

% =========================================================
\section{Limitations and Outlook}
\label{sec:limits}
The simulator is illustrative, not a physical model. Guarded execution (e.g., \texttt{unless} clauses) is preserved as comments but not yet lowered. Future work includes budgeted coherence, routing-cost hints, native back-ends (e.g., OPX), and block-level \texttt{floquet\{...\}} syntax.

% =========================================================
\section{Related Work}
\label{sec:related}
Pulse-level languages (e.g., \qua) and control compilers focus on waveform scheduling but do not encode semantic fields or defect dynamics. Domain-specific IRs and schedulers exist for hardware constraints; \SQUINT\ complements them by elevating semantics/topology to first-class compile-time entities.

% =========================================================
\section{Conclusion}
We introduced \SQUINT, a triple-aware compiler that makes semantics, topology, and thermodynamics explicit in source, validated at compile time, and reflected in pulse-level artifacts. The v0.1 toolchain demonstrates feasibility and offers a reproducible path from physical intent to scheduled controls.

\paragraph{Availability.} Source, examples, and docs: \url{https://github.com/Dr-Loo/SQUINT}.

\bibliographystyle{ieeetr}
\bibliography{refs}
\end{document}
